

%\documentclass{book}\begin{document}<content\end{document}
\documentclass[letter, 11pt, margins=0.3in]{texMemo}  % The texMemo package by Rob Oakes.

\usepackage{amsmath}
\usepackage{color, soul}
\usepackage{dcolumn}
    \newcolumntype{d}[1]{D{.}{.}{#1}}
\usepackage[colorlinks=true, citecolor=blue]{hyperref}
\usepackage{enumitem}
\usepackage{graphicx, lipsum}

\usepackage{caption}
\usepackage{subcaption}

\usepackage[LGRgreek]{mathastext}

\usepackage{listings}
\usepackage[]{matlab-prettifier}

\usepackage{multirow}

\usepackage{natbib}
\bibliographystyle{apalike}

\usepackage[table]{xcolor}
\definecolor{maroon}{cmyk}{0,0.87,0.68,0.32}
\definecolor{LightCyan}{rgb}{0.88,1,1}
\definecolor{Gray}{gray}{0.85}


\graphicspath{ {../} }


\memodate{\today~(Submitted)}
\memosubject{ACS 547, Noise Control Applications - Module 2 Assignment\\}



\begin{document}

\maketitle


\vspace{-0.25cm}
\section*{Problem 1 - Modal Behaviour of a Cylindrical Room}

The Matlab code for this problem is listed in Appendix~\ref{appendix:problem1}.

\vspace{0.25cm}
\subsection*{Problem 1a}

Table~\ref{table:resonanceFrequencies} lists the ten lowest resonance mode orders for the room and the respective frequency.

\setlength{\abovecaptionskip}{0pt}
\vspace{0.1cm}
{\renewcommand{\arraystretch}{1.5}
\begin{table}[h!]
    \begin{center}
        \small
        \begin{tabular}{ | c | c | c | }
            \hline
            \textbf{Index}  &  \textbf{Mode ($\mathbf n_x$, $\mathbf n_\theta$, $\mathbf n_r$)}  &  \textbf{Frequency [Hz]}  \\
            \hline
                0  &  0, 0, 0  &  0  \\
                \rowcolor{Gray}
                1  &  1, 0, 0  &  17.2  \\
                2  &  0, 1, 0  &  33.5  \\
                \rowcolor{Gray}
                3  &  2, 0, 0  &  34.3  \\
                4  &  1, 1, 0  &  37.6  \\
                \rowcolor{Gray}
                5  &  2, 1, 0  &  48.0  \\
                \rowcolor{pink}
                \hline
                6  &  3, 0, 0  &  51.5  \\
                \rowcolor{pink}
                \hline
                7  &  0, 2, 0  &  55.6  \\
                \hline
                8  &  1, 2, 0  &  58.2  \\
                \rowcolor{Gray}
                9  &  3, 1, 0  &  61.4  \\
                10  &  2, 2, 0  &  65.3  \\
            \hline
        \end{tabular}
    \end{center}
    \caption{Resonant modes of the cylindrical room.}
    \label{table:resonanceFrequencies}
\end{table}




\vspace{0.25cm}
\subsection*{Problem 1b}

The two closest modes are (3, 0, 0) and (0, 2, 0) with frequencies of 51.5 Hz and 55.6 Hz, respectively.




\vspace{0.25cm}
\subsection*{Problem 1c}

Figure~\ref{fig:modeVisualization} illustrates the (3, 0, 0) and (0, 2, 0) modes.  The white lines in each figure show the modal lines for that mode.  \textbf{The machine can be placed where the modal lines for each mode overlap}.  The figures were produced using the Room Eigenmode Simulator Version 1.1 software package.

The pink rings in Figure~\ref{fig:sub2} indicate 3 possible places where the machine could be placed.  These points coincide with the three modal planes shown in Figure~\ref{fig:sub1}.  Theoretically, there are an infinite number of places where the machine could be placed.  However, placement would take into account practical considerations such as accessibility, etc.


\begin{figure}[htbp]

    \centering
    
    \begin{subfigure}{0.5\textwidth}
        \includegraphics[width=\textwidth]{Mode Profile 51_5 Hz.png}
            \caption{Mode (3, 0, 0)}
            \label{fig:sub1}
    \end{subfigure}
      
    \vspace{0.25cm}
    \begin{subfigure}{0.5\textwidth}
        \includegraphics[width=\textwidth]{Mode Profile 55_8 Hz.png}
            \caption{Mode (0, 2, 0)}
            \label{fig:sub2}
    \end{subfigure}
    
        \vspace{0.5cm}
    \caption{Visualization of modes.  (a.) Mode (3, 0, 0).  (b.) Model (0, 2, 0).  The pink circles in~\ref{fig:sub2} illustrate a few modal line intersections where the machine could be placed.}
    \label{fig:modeVisualization}
      
\end{figure}






\newpage
\section*{Problem 2 - Sabine Room}

The Matlab code for this problem is listed in Appendix~\ref{appendix:problem2}.


\vspace{0.25cm}
\subsection*{Problem 2a}

The reverberant field sound pressure level is approximately 98.3 dB SPL.




\vspace{0.25cm}
\subsection*{Problem 2b}

Figure~\ref{figure:sabineRoomLevels} shows the direct, reverberant, and total sound pressure levels for a 25 mW, 125 Hz, broadband, omnidirectional source placed centrally in the room (i.e. the directivity factor is 1).

\vspace{-4cm}
\begin{figure}[htb!]

    \center
        \includegraphics[ scale = 0.65, keepaspectratio ]{Sabine Room.pdf}

    \vspace{-3.5cm}
    \caption{Sound levels for the room produced by a 25 mW, 125 Hz omnidirectional, broadband source.}
    \label{figure:sabineRoomLevels}

\end{figure}






\newpage
\section*{Problem 3 - Transmission Loss Measurement}

The Matlab code for this problem is listed in Appendix~\ref{appendix:problem3}.


Figure~\ref{figure:q3AverageAbsorption} shows the average absorption per octave band based on the T60 data.  The calibration plate isolates the receiver room and the absorption calculation does not consider the calibration plate.


\begin{figure}[htb!]

    \center
        \includegraphics[ scale = 0.6, keepaspectratio ]{Q3 Average Absorption.png}

    \caption{Average absorption per octave band.}
    \label{figure:q3AverageAbsorption}

\end{figure}


\vspace{0.2cm}
Figure~\ref{figure:q3TransmissionLoss} shows the transmission loss per octave band.  


\begin{figure}[htb!]

    \center
        \includegraphics[ scale = 0.6, keepaspectratio ]{Q3 Transmission Loss.png}

    \caption{Transmission loss per octave band.}
    \label{figure:q3TransmissionLoss}

\end{figure}






\newpage
\section*{Problem 4 - Panel Transmission Loss}

The Matlab code for this problem is listed in Appendix~\ref{appendix:problem4}.

\subsection*{Problem 4a}

Table~\ref{table:machNumbers} lists the Mach numbers for each pipe section.  The flow rate is 0.017462 $\frac{m^3}{s}$.

\setlength{\abovecaptionskip}{0pt}
\vspace{0.1cm}
{\renewcommand{\arraystretch}{1.5}
\begin{table}[h!]
    \begin{center}
        \small
        \begin{tabular}{ | c | c | c | }
            \hline
            \textbf{Pipe}  &  \textbf{Area [}{$\mathbf m^2$}\textbf{]}  &  \textbf{Mach Number [unitless]}  \\
            \hline
                Inlet  &  0.000507  &  -0.10047  \\
                \hline
                \rowcolor{Gray}
                Outlet  &  0.00811  &  -0.0062795  \\
            \hline
        \end{tabular}
    \end{center}
    \caption{Calculated Mach numbers.}
    \label{table:machNumbers}
\end{table}



\subsection*{Problem 4b}

\vspace{0.25cm}
Figure~\ref{figure:intakeSystem} shows the transmission loss profiles.

%\vspace{0.25cm}
%\begin{figure}[!htb]
%    \center
%        \includegraphics[ scale = 0.6, keepaspectratio ]{Intake System.pdf}
%    \caption{Transmission loss profiles for no flow, flow, and flow with a lossy Helmholtz resonator.}
%    \label{figure:intakeSystem}
%\end{figure}

\vspace{0.25cm}
The addition of flow to the intake system introduces a slight phase delay, a lower overall level of loss (approximately 22 dB), and greater loss at the dips.  The phase delay is easier to see at respective dips in the loss profile.



\vspace{0.25cm}
\subsubsection*{i - Critical Frequency and Coincidence Frequency at 75$^\circ$}

ph



\vspace{0.25cm}
\subsubsection*{ii - Transmission Loss at Angle of Incidence of 75$^\circ$}

ph



\vspace{0.25cm}
\subsubsection*{iii - Transmission Loss for Angles of Incidence between 0-90$^\circ$}

ph



\vspace{0.25cm}
\subsubsection*{iv - Diffuse Transmission Loss}

ph



\newpage
\subsection*{Problem 4c}



\subsection*{Problem 4d}

\vspace{0.25cm}
\subsubsection*{i - Critical Frequency and Coincidence Frequency at 75$^\circ$}

ph

\vspace{0.25cm}
\subsubsection*{ii - Transmission Loss at Angle of Incidence of 75$^\circ$}

ph

vspace{0.25cm}
\subsubsection*{iii - Transmission Loss for Angles of Incidence between 0-90$^\circ$}

ph

\vspace{0.25cm}
\subsubsection*{iv - Diffuse Transmission Loss}





%\vspace{0.25cm}
%Figure~\ref{figure:intakeSystem} shows the transmission loss profiles.


%\newpage
%Table~\ref{table:helmholzResonator} lists the dimensions of the lossy Helmholtz resonator for the 129 Hz notch.
%
%\setlength{\abovecaptionskip}{0pt}
%\vspace{0.1cm}
%{\renewcommand{\arraystretch}{1.5}
%\begin{table}[h!]
%    \begin{center}
%        \small
%        \begin{tabular}{ | l | c | }
%            \hline
%            \textbf{Item}  &  \textbf{Measure]}  \\
%            \hline
%                Cavity Diameter                 &  0.254 m  \\
%                \hline
%                \rowcolor{Gray}
%                Neck Diameter                   &  0.0254 m  \\
%                \hline
%                Neck Length                     &  0.127 m  \\
%                \hline
%                \rowcolor{Gray}
%                Neck Area                       &  0.5e-3 $m^2$  \\
%                \hline
%                Length Correction 1, $L_{o1}$   &  0.711 m  \\
%                \hline
%                \rowcolor{Gray}
%                Length Correction 2, $L_{o2}$   &  0.3 m  \\
%                \hline
%                Cavity Volume                   &  8.0e-5 $m^3$  \\
%                \hline
%                \rowcolor{Gray}
%                Q Factor   &  10  \\
%            \hline
%        \end{tabular}
%    \end{center}
%    \caption{Dimensions of the lossy Helmholtz resonator (see slide 15 of Lecture 3 notes).}
%    \label{table:helmholzResonator}
%\end{table}










\newpage
\section*{Problem 5 - Large Enclosure Design}

The Matlab code for this problem is listed in Appendix~\ref{appendix:problem5}.

\vspace{0.25cm}
ph







\newpage
\section*{Problem 6 - Close-fitting Enclosure Design}

The Matlab code for this problem is listed in Appendix~\ref{appendix:problem5}.

\vspace{0.25cm}
ph






\newpage
\section{Appendix - Matlab Code for Problem 1}
\label{appendix:problem1}

\lstinputlisting[style=Matlab-editor, numbers=left, mlshowsectionrules, basicstyle=\scriptsize]{../ACS_547_Module_2_Question_1_Wednesday_February_12_2025.m}


\newpage
\section{Appendix - Matlab Code for Problem 2}
\label{appendix:problem2}

\lstinputlisting[style=Matlab-editor, numbers=left, mlshowsectionrules, basicstyle=\scriptsize]{../ACS_547_Module_2_Question_2_Wednesday_February_12_2025.m}


\newpage
\section{Appendix - Matlab Code for Problem 3}
\label{appendix:problem3}

\lstinputlisting[style=Matlab-editor, numbers=left, mlshowsectionrules, basicstyle=\scriptsize]{../ACS_547_Module_2_Question_3_Wednesday_February_12_2025.m}


\newpage
\section{Appendix - Matlab Code for Problem 4}
\label{appendix:problem4}

\lstinputlisting[style=Matlab-editor, numbers=left, mlshowsectionrules, basicstyle=\scriptsize]{../ACS_547_Module_2_Question_4_Wednesday_February_12_2025.m}


\newpage
\section{Appendix - Matlab Code for Problem 5}
\label{appendix:problem5}

\lstinputlisting[style=Matlab-editor, numbers=left, mlshowsectionrules, basicstyle=\scriptsize]{../ACS_547_Module_2_Question_5_Wednesday_February_12_2025.m}


\newpage
\section{Appendix - Matlab Code for Problem 6}
\label{appendix:problem5}

\lstinputlisting[style=Matlab-editor, numbers=left, mlshowsectionrules, basicstyle=\scriptsize]{../ACS_547_Module_2_Question_6_Wednesday_February_12_2025.m}






\end{document}


% Reference(s):

% https://tex.stackexchange.com/questions/180222/how-to-change-font-size-for-specific-lstlisting


































