

%\documentclass{book}\begin{document}<content\end{document}
\documentclass[letter, 11pt, margins=0.5in]{texMemo}  % The texMemo package by Rob Oakes.

\usepackage{amsmath}
\usepackage[colorlinks=true, citecolor=blue]{hyperref}
\usepackage{enumitem}
\usepackage{graphicx, lipsum}

\usepackage[LGRgreek]{mathastext}

\usepackage{matlab-prettifier}

\usepackage{natbib}
\bibliographystyle{apalike}


\memodate{\today~(Submitted)}
%\memoto{Dr. Dan Russell - College of Engineering, Penn State University}
%\memofrom{Michael R. Wirtzfeld - Sound Discovery LLC\\}
\memosubject{Noise Control Applications - Module 1 Assignement\\}

%\memologo{\includegraphics[width=1.0\textwidth]{SOUND_DISCOVERY_Business_Card.jpg}}



\begin{document}

\maketitle


\section*{Problem 1 - Cut-on Frequencies in Ducts and Pipes}

\subsection*{Problem 1a}

The lowest cut-on frequency for a rectangular duct with air flow is given by equation,

\vspace{-0.25cm}
\begin{equation}
    f_{cut-on} = 0.5 \cdot \frac{c}{L}
\end{equation}

where $c$ is the speed of sound in air, 343~$\frac{m}{s}$,  and $L$ is the largest side of the rectangular cross-section.

\vspace{0.25cm}
With cross-sectional dimensions of $L_x = 12~cm$ and $L_y = 20~cm$, the lowest cut-on frequency for this rectangular duct is,

\vspace{-0.25cm}
\begin{equation*}
    f_{cut-on} = 0.5 \cdot \frac{ 343~\frac{m}{s} }{ 0.20~m } = \boldsymbol{857.5~Hz}
\end{equation*}




\subsection*{Problem 1b}

The lowest cut-on frequency for a circular duct with air flow with the same cross-sectional area as the rectangular duct in part (a.) can be calculated using equation,

\vspace{-0.25cm}
\begin{equation}
    f_{cut-on} = 0.568 \cdot \frac{c}{d}
    \label{equation:circularDuct}
\end{equation}

where $c$ is the speed of sound in air, 343~$\frac{m}{s}$,  and $d$ is diameter of the circular duct.

\vspace{0.25cm}
The cross-sectional area of the rectangular duct is,

\vspace{-0.25cm}
\begin{equation*}
    Area_{~rectangular~duct} = 0.12~m~\cdot0.20~m =  0.024~m
\end{equation*}

\vspace{0.25cm}
The corresponding diameter for this area is,

\vspace{-0.25cm}
\begin{equation*}
    diameter = \sqrt{ \frac{0.24~m^2}{\pi} } \cdot 2 = 0.17~m
\end{equation*}

\vspace{0.25cm}
Using Eq.~\ref{equation:circularDuct}, the lowest cut-on frequency for this circular duct with air flow is,

\vspace{-0.25cm}
\begin{equation*}
    f_{cut-on} = 0.568 \cdot \frac{ 1,500~\frac{m}{s} }{ 0.17~m } = \boldsymbol{1,114.5~Hz}
\end{equation*}





\subsection*{Problem 1c}

The lowest cut-on frequency for this circular duct with water flow can be calculated using Eq.~\ref{equation:circularDuct},

\vspace{-0.25cm}
\begin{equation*}
    f_{cut-on} = 0.568 \cdot \frac{ 1,500~\frac{m}{s} }{ 0.17~m } = \boldsymbol{4,873.9~Hz}
\end{equation*}

The lowest cut-on frequency for water is considerable larger than it is for air flow.





\newpage
\subsection*{Problem 1d}

The speed of sound in air is calculated by,

\vspace{-0.25cm}
\begin{equation}
    c = \sqrt{ \gamma \cdot R \cdot T_K }
    \label{equation:speedOfSoundInAir}
\end{equation}

where $\gamma = 1.4$ is the ratio of specific heats, $R = 287~\frac{J}{kg \cdot K}$ is the gas constant, and $T_K$ is the absolute temperature in Kelvin.

\vspace{0.25cm}
Figure~\ref{figure:cuton_frequency_versus_temperature} illustrates how the lowest cut-on frequency changes as the air heats from $0^{\circ}$ to $500^{\circ}$ Celsius.

\vspace{0.25cm}
The square-root relationship between temperature and the speed of sound in air is apparent and governs the behaviour of the cut-on frequency.




\vspace{-4.8cm}
\begin{figure}[!htb]

    \center
        \includegraphics[ scale = 0.8, keepaspectratio ]{Cut-on Frequency Versus Temperature - Sunday, January 19, 2025.pdf}

    \vspace{-5.2cm}
    \caption{Lowest cut-on frequency for a circular 5 cm diameter duct versus air temperature.}
    \label{figure:cuton_frequency_versus_temperature}

\end{figure}






\subsection*{Problem 1e}

 \textbf{Question:  Are cut-on frequencies higher for a circular or rectangular duct for a given cross-sectional area?}

 The lowest cut-on frequency is higher for a circular duct than for a
 rectangular duct for a given cross-sectional area.

 For the dimensions given in class, the rectangular duct is not square.
 This produces a larger dimension and thus a smaller, lowest cut-on
 frequency.  If the rectangular duct is square dimensions on the order
 of the circular duct diameter with the same cross-sectional area, the
 cut-on frequencies are approximately equal.



\vspace{0.25cm}
\textbf{Question:  What about in air versus water?}

 The lowest cut-on frequency is larger for water than for air.  The cut-on
 frequency is proportional to the speed of sound and the speed of sound in
 water is greater than the speed of sound in air.



\vspace{0.25cm}
\textbf{Question:  What about cold versus hot air?}

 The lowest cut-on frequency is higher for warm air than it is for cold air.









\newpage
\section*{Problem 2 - Muffler Design Comparison}

For these muffler comparisons, it is assumed that there is no resistive terms (i.e., damping) and no flow.  Since transmission losses are plotted, the end corrections (i.e., load impedance at outlet of the system) have no physical meaning and are not accounted for in the computations.

   
\subsection*{Problem 2a}

Figure~\ref{figure:problem2figure1} shows the transmission loss profile for the simple expansion chamber (red, dashed line).





Peak occur at a quarter wavelength;  extension tube at outlet.

Extension tube aids quarter wavelenths.


\begin{figure}[!htb]

    \center
        \includegraphics[ scale = 0.6, keepaspectratio ]{Assignment 1 - Question 2 Figure All TL Profiles.pdf}

    \caption{Transmission loss profiles for a simple expansion chamber, a double-tuned expansion chamber, and a cascaded double-tuned expansion chamber mufflers.}
    \label{figure:problem2figure1}

\end{figure}


\begin{figure}[!htb]

    \center
        \includegraphics[ scale = 0.6, keepaspectratio ]{Assignment 1 - Question 2 Figure Comparison TL Plot For Cascaded Systems.pdf}

    \caption{Transmission loss profiles for a cascaded double-tuned expansion chamber muffler and a modified, cascaded double-tuned expansion chamber mufflers.}
    \label{figure:problem2figure2}

\end{figure}





\subsection*{Problem 2b}

There is no damping in the system;  resonances will be artificially high.

\subsection*{Problem 2c}

There is no damping in the system;  resonances will be artificially high.










\newpage
\section*{Problem 3 - Bugle Recorder}

Diameters of holes should be smaller than a wavelength.

$R_A$ is neglected (energy loss).

\subsection*{Problem 3a}

\subsection*{Problem 3b}










\newpage
\section*{Problem 4 - Intake Duct}

\subsection*{Problem 4a}

\subsection*{Problem 4b}

\subsection*{Problem 4c}

\subsection*{Problem 4d}










\newpage
\section*{Problem 5 - Intake Duct Silencer}

\subsection*{Problem 5a}

\subsection*{Problem 5b}

\subsection*{Problem 5c}

\subsection*{Problem 5d}

\subsection*{Problem 5e}






\newpage
\section*{Appendix A:  Matlab Code - Problem 1}

\begin{lstlisting}[style=Matlab-editor, basicstyle=\mlttfamily, numbers=none, keepspaces, mlshowsectionrules]
%% Synopsis

% Homework Set 1 - Cut-on Frequencies in Ducts and Pipes

% Note:  Send draft of report before submission for comments.
%
% Dimensions are annontated in the class notes.



%% To Do

% Focus on interpretation.



%% Environment

close all; clear; clc;
% restoredefaultpath;

% set( 0, 'DefaultFigurePosition', [  400  400  900  400  ] );  % [ left bottom width height ]
set( 0, 'DefaultFigurePaperPositionMode', 'manual' );
set( 0, 'DefaultFigureWindowStyle', 'normal' );
set( 0, 'DefaultLineLineWidth', 1.5 );
set( 0, 'DefaultTextInterpreter', 'Latex' );

format ShortG;

pause( 1 );

PRINT_FIGURES = 0;



%% Define Values and Functions

c_air = 343;  % The speed of sound in air in meters per second.
c_water = 1500;  % The speed of sound in water in meters per second.

gamma = 1.4;  % The ratio of specific heats [unitless].
R = 287;  % The gas constant [Joules per ( kilogram * Kelvin)].


h_f_cut_on_rectangular_duct = @( c, L )  0.5 .* c ./ L;
%
% c - The speed of sound.
% L - The largest cross-section dimension of the rectangular duct.


h_f_cut_on_circular_duct = @( c, d )  0.568 .* c ./ d;
%
% c - The speec of sound.
% L - The diameter of the circular duct.


h_speed_of_sound_in_air = @( gamma, R, temperature_Kelvin)  sqrt( gamma .* R .* temperature_Kelvin );



%% Problem 1a

% The cross-sectional dimensions for the rectangular duct are:  Lx = 12 cm and Ly = 20 cm.

% The largest dimension is Ly = 20 cm or 0.2 m.

% The cut-on frequency is,
h_f_cut_on_rectangular_duct( c_air, 0.2 );  % 857.5 Hz (shown in class 858 Hz)
    fprintf( 1, '\n Problem 1a:  The lowest cut-on frequency for the rectangular pipe with air is %3.1f Hz.\n', h_f_cut_on_rectangular_duct( c_air, 0.2 ) );



%% Problem 1b

% The cross-sectional dimensions for the rectangular duct are:  Lx = 12 cm and Ly = 20 cm.

% The cross-sectional area of the rectangular duct is 12 cm * 20 cm = 240 cm^2 or 0.024 m^2.
rectangular_duct_cross_sectional_area = 0.12 * 0.20;  % 0.024 m^2

% The diameter of the circulat pipe is,
circular_duct_diameter = sqrt( 0.024 / pi ) * 2;  % 0.17481 meters
%
% Check:
    % pi * ( circular_duct_diameter / 2 )^2  CHECKED


% The cut-on frequency for the circular duct is,
h_f_cut_on_circular_duct( c_air, circular_duct_diameter );  % 1,114.5 Hz
    fprintf( 1, '\n Problem 1b:  The lowest cut-on frequency for the circular pipe (of equal area) with air is %3.1f Hz.\n', h_f_cut_on_circular_duct( c_air, circular_duct_diameter ) );



%% Problem 1c

% The cut-on frequency for the circular duct with water is,
h_f_cut_on_circular_duct( c_water, circular_duct_diameter );  % 4,873.9 Hz
    fprintf( 1, '\n Problem 1c:  The lowest cut-on frequency for the circular pipe (of equal area) with water is %3.1f Hz.\n', h_f_cut_on_circular_duct( c_water, circular_duct_diameter ) );

% The cut-on frequency should be higher because it is proportional to the
% speed of sound in a given medium.



%% Problem 1d

fprintf( 1, '\n Problem 1d:  See the figure.\n' );

temperature_range_celsius = 0:0.1:500;  % Celsius
    temperature_range_kelvin = temperature_range_celsius + 273.15;  % Kelvin


FONT_SIZE = 14;

figure( ); ...
    plot( temperature_range_celsius, h_f_cut_on_circular_duct( h_speed_of_sound_in_air( gamma, R, temperature_range_kelvin ), 0.05 ) ./ 1e3 );  grid on;
        legend( 'Duct Diameter = 5.0 cm', 'Location', 'East', 'FontSize', FONT_SIZE, 'Interpreter', 'Latex' );
        set( gca, 'FontSize', FONT_SIZE );
    %
    xlabel( 'Temperature [Celsius]', 'FontSize', FONT_SIZE );
        % xl = get( gca, 'xlabel' );    pxl = get( xl, 'position' );  pxl( 2 ) = 1.1 * pxl( 2 );
        %     set( xl, 'position', pxl );
    %
    ylabel( 'Lowest Cut-on Frequency [kHz]', 'FontSize', FONT_SIZE );
        % yl = get( gca, 'ylabel' );  pyl = get( yl, 'position' );  pyl( 1 ) = 1.2 * pyl( 1 );
        %     set( yl, 'position', pyl );
    %
    caption = sprintf( 'Lowest Cut-on Frequency for a Circular Pipe with Air Flow Versus Air Temperature\n' );
        title( caption, 'FontSize', FONT_SIZE );
    %
    ylim( [ 3  7 ] );


% if ( PRINT_FIGURES == 1 )
%     saveas( gcf, 'Cut-on Frequency Versus Temperature - Sunday, January 19, 2025.pdf' );
% end



%% Problem 1e

fprintf( 1, '\n Problem 1e:  See Section Problem 1e of the Matlab script for the answers.\n\n' );



%% Clean-up

if ( ~isempty( findobj( 'Type', 'figure' ) ) )
    monitors = get( 0, 'MonitorPositions' );
        if ( size( monitors, 1 ) == 1 )
            autoArrangeFigures( 2, 2, 1 );
        elseif ( 1 < size( monitors, 1 ) )
            autoArrangeFigures( 2, 2, 1 );
        end
end


if ( PRINT_FIGURES == 1 )
    saveas( gcf, 'Cut-on Frequency Versus Temperature - Sunday, January 19, 2025.pdf' );
end


fprintf( 1, '\n\n\n*** Processing Complete ***\n\n\n' );



%% Reference(s)
\end{lstlisting}








\end{document}


































