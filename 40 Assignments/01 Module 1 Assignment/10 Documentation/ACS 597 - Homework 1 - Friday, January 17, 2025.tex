

%\documentclass{book}\begin{document}<content\end{document}
\documentclass[letter, 11pt, margins=0.25in]{texMemo}  % The texMemo package by Rob Oakes.

\usepackage{amsmath}
\usepackage{color, soul}
\usepackage{dcolumn}
    \newcolumntype{d}[1]{D{.}{.}{#1}}
\usepackage[colorlinks=true, citecolor=blue]{hyperref}
\usepackage{enumitem}
\usepackage{graphicx, lipsum}

\usepackage[LGRgreek]{mathastext}

\usepackage{matlab-prettifier}
\usepackage{multirow}

\usepackage{natbib}
\bibliographystyle{apalike}

\usepackage[table]{xcolor}
\definecolor{maroon}{cmyk}{0,0.87,0.68,0.32}
\definecolor{LightCyan}{rgb}{0.88,1,1}
\definecolor{Gray}{gray}{0.85}


\memodate{\today~(Submitted)}
%\memoto{Dr. Dan Russell - College of Engineering, Penn State University}
%\memofrom{Michael R. Wirtzfeld - Sound Discovery LLC\\}
\memosubject{ACS 547, Noise Control Applications - Module 1 Assignment\\}

%\memologo{\includegraphics[width=1.0\textwidth]{SOUND_DISCOVERY_Business_Card.jpg}}



\begin{document}

\maketitle


\vspace{-0.25cm}
\section*{Problem 1 - Cut-on Frequencies in Ducts and Pipes}

The Matlab code for this problem is listed in Appendix~\ref{appendix:problem1}.

\vspace{-0.25cm}
\subsection*{Problem 1a}

The lowest cut-on frequency for a rectangular duct with air flow is given by equation,

\vspace{-0.25cm}
\begin{equation}
    f_{cut-on} = 0.5 \cdot \frac{c}{L}
\end{equation}

where $c$ is the speed of sound in air, 343~$\frac{m}{s}$,  and $L$ is the largest side of the rectangular cross-section.

\vspace{0.25cm}
With cross-sectional dimensions of $L_x = 12~cm$ and $L_y = 20~cm$, the lowest cut-on frequency for this rectangular duct is,

\vspace{-0.25cm}
\begin{equation*}
    f_{cut-on} = 0.5 \cdot \frac{ 343~\frac{m}{s} }{ 0.20~m } = \boldsymbol{857.5~Hz}
\end{equation*}



\vspace{-0.25cm}
\subsection*{Problem 1b}

The lowest cut-on frequency for a circular duct with air flow with the same cross-sectional area as the rectangular duct in part (a.) can be calculated using equation,

\vspace{-0.25cm}
\begin{equation}
    f_{cut-on} = 0.568 \cdot \frac{c}{d}
    \label{equation:circularDuct}
\end{equation}

where $c$ is the speed of sound in air, 343~$\frac{m}{s}$,  and $d$ is diameter of the circular duct.

\vspace{0.25cm}
The cross-sectional area of the rectangular duct is,

\vspace{-0.25cm}
\begin{equation*}
    Area_{~rectangular~duct} = 0.12~m~\cdot0.20~m =  0.024~m
\end{equation*}

\vspace{0.25cm}
The corresponding diameter for this area is,

\vspace{-0.25cm}
\begin{equation*}
    diameter = \sqrt{ \frac{0.24~m^2}{\pi} } \cdot 2 = 0.17~m
\end{equation*}

\vspace{0.25cm}
Using Eq.~\ref{equation:circularDuct}, the lowest cut-on frequency for this circular duct with air flow is,

\vspace{-0.25cm}
\begin{equation*}
    f_{cut-on} = 0.568 \cdot \frac{ 1,500~\frac{m}{s} }{ 0.17~m } = \boldsymbol{1,114.5~Hz}
\end{equation*}




\vspace{-0.25cm}
\subsection*{Problem 1c}

The lowest cut-on frequency for this circular duct with water flow can be calculated using Eq.~\ref{equation:circularDuct},

\vspace{-0.25cm}
\begin{equation*}
    f_{cut-on} = 0.568 \cdot \frac{ 1,500~\frac{m}{s} }{ 0.17~m } = \boldsymbol{4,873.9~Hz}
\end{equation*}

The lowest cut-on frequency for water is considerable larger than it is for air flow.





\vspace{-0.25cm}
\subsection*{Problem 1d}

The speed of sound in air is calculated by,

\vspace{-0.25cm}
\begin{equation}
    c = \sqrt{ \gamma \cdot R \cdot T_K }
    \label{equation:speedOfSoundInAir}
\end{equation}

where $\gamma = 1.4$ is the ratio of specific heats, $R = 287~\frac{J}{kg \cdot K}$ is the gas constant, and $T_K$ is the absolute temperature in Kelvin.

\vspace{0.25cm}
Figure~\ref{figure:cuton_frequency_versus_temperature} illustrates how the lowest cut-on frequency changes as the air heats from $0^{\circ}$ to $500^{\circ}$ Celsius.

\vspace{0.25cm}
The square-root relationship between temperature and the speed of sound in air is apparent and governs the behaviour of the cut-on frequency.




\vspace{-4.8cm}
\begin{figure}[!htb]

    \center
        \includegraphics[ scale = 0.8, keepaspectratio ]{Cut-on Frequency Versus Temperature - Sunday, January 19, 2025.pdf}

    \vspace{-5.2cm}
    \caption{Lowest cut-on frequency for a circular 5 cm diameter duct versus air temperature.}
    \label{figure:cuton_frequency_versus_temperature}

\end{figure}





\vspace{-0.25cm}
\subsection*{Problem 1e}

 \textbf{Question:  Are cut-on frequencies higher for a circular or rectangular duct for a given cross-sectional area?}

 The lowest cut-on frequency is higher for a circular duct than for a
 rectangular duct for a given cross-sectional area.

 For the dimensions given in class, the rectangular duct is not square.
 This produces a larger dimension and thus a smaller, lowest cut-on
 frequency.  If the rectangular duct is square dimensions on the order
 of the circular duct diameter with the same cross-sectional area, the
 cut-on frequencies are approximately equal.



\vspace{0.25cm}
\textbf{Question:  What about in air versus water?}

 The lowest cut-on frequency is larger for water than for air.  The cut-on
 frequency is proportional to the speed of sound and the speed of sound in
 water is greater than the speed of sound in air.



\vspace{0.25cm}
\textbf{Question:  What about cold versus hot air?}

 The lowest cut-on frequency is higher for warm air than it is for cold air.









\newpage
\section*{Problem 2 - Muffler Design Comparison}

The Matlab code for this problem is listed in Appendix~\ref{appendix:problem2}.

For these muffler comparisons, the following assumptions were made:

\begin{itemize}
  \item There is no flow.
  \item There are no resistive terms.
  \item The load impedance was not included because the transmission loss does not require it.
\end{itemize}


\vspace{-0.25cm}
\subsection*{Problems 2a, 2b, and 2c}

\vspace{-0.25cm}
Figure~\ref{figure:problem2figure1} shows the transmission loss profiles for a simple expansion chamber, a double-tuned expansion chamber, and a cascaded double-tuned expansion chamber muffler.

The peaks for the simple expansion chamber (red, dashed line) are approximately 22 dB and occur at frequencies with a wavelength that is a quarter of the length of the expansion chamber.  Minimal loss occurs at half wavelength multiples.

The addition of the extension tube inside the muffler produces a quarter wavelength resonator.  The side branch of Ji (2005;  Slide 11, Lecture 3 notes) was used to calculate $L_o$.  For the cascaded double-tuned expansion chamber, the extension tubes produce a secondary quarter wavelength resonator.

As noted in office hours, there is no damping which produces artificially high resonances.


\begin{figure}[!htb]

    \center
        \includegraphics[ scale = 0.675, keepaspectratio ]{Assignment 1 - Question 2 Figure All TL Profiles.pdf}

    \caption{Transmission loss profiles for a simple expansion chamber, a double-tuned expansion chamber, and a cascaded double-tuned expansion chamber muffler.}
    \label{figure:problem2figure1}

\end{figure}



\subsection*{Problem 2d}

Figure~\ref{figure:problem2figure2} shows the transmission loss profiles for a cascaded double-tuned expansion chamber and a modified version of this muffler.

\newpage
Two modifications were made to the original muffler:

\begin{enumerate}
  \item The left 3" extension tube in the left chamber was shortened to 2" inches, making the respective muffler section 1" longer.
  \item The left 3" extension tube in the right chamber was lengthened to 4", making the respective muffler section 1" shorter.
\end{enumerate}

\vspace{0.5cm}
\begin{figure}[!htb]

    \center
        \includegraphics[ scale = 0.675, keepaspectratio ]{Assignment 1 - Question 2 Figure Comparison TL Plot For Cascaded Systems.pdf}

    \caption{Transmission loss profiles for a cascaded double-tuned expansion chamber muffler and a modified version of this muffler.}
    \label{figure:problem2figure2}

\end{figure}

\vspace{0.25cm}
These modifications change the symmetry of the cascaded system, and allow the resonate frequencies to be changed independently.












\newpage
\section*{Problem 3 - Bugle Recorder}


\vspace{-0.25cm}
\subsection*{Problem 3a}

The Matlab code for this problem is listed in Appendix~\ref{appendix:problem3}.

Table~\ref{table:mouthpieceAndPip} lists the length of the pipe section and the mouthpiece.

\setlength{\abovecaptionskip}{0pt}
\vspace{0.1cm}
{\renewcommand{\arraystretch}{1.5}
\begin{table}[h!]
    \begin{center}
        \small
        \begin{tabular}{ | c | c | }
            \hline
            \textbf{Item}  &  \textbf{Length [mm]}  \\
            \hline
                Pipe  &  145  \\
                \hline
                \rowcolor{Gray}
                Mouthpiece  &  90  \\
            \hline
        \end{tabular}
    \end{center}
    \caption{Calculated length of the pipe and length of the mouthpiece.}
    \label{table:mouthpieceAndPip}
\end{table}


\vspace{-0.25cm}
\subsection*{Problem 3b}

Table~\ref{table:holePlacementSummary} summarizes the placement of the holes for each note relative to the end of the pipe.

\setlength{\abovecaptionskip}{0pt}
\vspace{0.1cm}
{\renewcommand{\arraystretch}{1.5}
\begin{table}[h!]
    \begin{center}
        \small
        \begin{tabular}{ | c | c | c | }
            \hline
            \textbf{Note}  &  \textbf{Frequency [Hz]}  &  \textbf{Length from End of Pipe [mm]}  \\
            \hline
                C5  &  523      &  n/a  \\
                \rowcolor{Gray}
                F5  &  698      &  87.75  \\
                A5  &  880      &  138.25  \\
                \rowcolor{Gray}
                C6  &  1,046    &  0.168  \\
            \hline
        \end{tabular}
    \end{center}
    \caption{Hole placement distances.}
    \label{table:holePlacementSummary}
\end{table}


\vspace{0.25cm}
Figure~\ref{figure:bugleRecorderNoteSpectrums} shows the respective spectrum for each of the bugle recorder notes.

\vspace{0.25cm}
\begin{figure}[!htb]
    \center
        \includegraphics[ scale = 0.6, keepaspectratio ]{Assignment 1 - Question 3 Bugle Recorder Note Spectrums.pdf}
    \caption{Spectrum for the C5, F5, A5, and C6 bugle recorder notes.}
    \label{figure:bugleRecorderNoteSpectrums}
\end{figure}












\newpage
\section*{Problem 4 - Intake Duct}

The Matlab code for this problem is listed in Appendix~\ref{appendix:problem4}.

\subsection*{Problem 4a}

Table~\ref{table:machNumbers} lists the Mach numbers for each pipe section.  The flow rate is 0.017462 $\frac{m^3}{s}$.

\setlength{\abovecaptionskip}{0pt}
\vspace{0.1cm}
{\renewcommand{\arraystretch}{1.5}
\begin{table}[h!]
    \begin{center}
        \small
        \begin{tabular}{ | c | c | c | }
            \hline
            \textbf{Pipe}  &  \textbf{Area [}{$\mathbf m^2$}\textbf{]}  &  \textbf{Mach Number [unitless]}  \\
            \hline
                Inlet  &  0.000507  &  -0.10047  \\
                \hline
                \rowcolor{Gray}
                Outlet  &  0.00811  &  -0.0062795  \\
            \hline
        \end{tabular}
    \end{center}
    \caption{Calculated Mach numbers.}
    \label{table:machNumbers}
\end{table}



\subsection*{Problems 4b, 4c, and 4d}

\vspace{0.25cm}
Figure~\ref{figure:intakeSystem} shows the transmission loss profiles.

\vspace{0.25cm}
\begin{figure}[!htb]
    \center
        \includegraphics[ scale = 0.6, keepaspectratio ]{Intake System.pdf}
    \caption{Transmission loss profiles for no flow, flow, and flow with a lossy Helmholtz resonator.}
    \label{figure:intakeSystem}
\end{figure}

\vspace{0.25cm}
The addition of flow to the intake system introduces a slight phase delay, a lower overall level of loss (approximately 22 dB), and greater loss at the dips.  The phase delay is easier to see at respective dips in the loss profile.

The phase delay is introduced by each of the pipe segments by the complex-exponential term in the respective transfer function.  The attenuation is introduced by the expansion transfer function.


\newpage
Table~\ref{table:helmholzResonator} lists the dimensions of the lossy Helmholtz resonator for the 129 Hz notch.

\setlength{\abovecaptionskip}{0pt}
\vspace{0.1cm}
{\renewcommand{\arraystretch}{1.5}
\begin{table}[h!]
    \begin{center}
        \small
        \begin{tabular}{ | l | c | }
            \hline
            \textbf{Item}  &  \textbf{Measure]}  \\
            \hline
                Cavity Diameter                 &  0.254 m  \\
                \hline
                \rowcolor{Gray}
                Neck Diameter                   &  0.0254 m  \\
                \hline
                Neck Length                     &  0.127 m  \\
                \hline
                \rowcolor{Gray}
                Neck Area                       &  0.5e-3 $m^2$  \\
                \hline
                Length Correction 1, $L_{o1}$   &  0.711 m  \\
                \hline
                \rowcolor{Gray}
                Length Correction 2, $L_{o2}$   &  0.3 m  \\
                \hline
                Cavity Volume                   &  8.0e-5 $m^3$  \\
                \hline
                \rowcolor{Gray}
                Q Factor   &  10  \\
            \hline
        \end{tabular}
    \end{center}
    \caption{Dimensions of the lossy Helmholtz resonator (see slide 15 of Lecture 3 notes).}
    \label{table:helmholzResonator}
\end{table}

The addition of the Helmholtz resonator offset the attenuation of the dip to approximately 0 dB.










\newpage
\section*{Problem 5 - Intake Duct Silencer}

The Matlab code for this problem is listed in Appendix~\ref{appendix:problem5}.

My respective dip occurs at approximately 1,060 Hz, which has a loss of -25 dB.  \textbf{For this problem I used the 940 Hz dip noted in the assignment discussion, requiring a 14 dB correction.}

\subsection*{Problem 5a}

As noted in the discussion, there are two cases to be considered here.

First, with a liner thickness of 0.0381 meters and a half-liner width (circular duct) of 0.0113 meters, the expansion ratio, $m$, is 2.7.  From Figure 8.37, a 14 dB overall loss has a total liner attenuation of 10 dB.  The attenuation rate is about 78.4 $\frac{dB}{m}$.

Second,

% Figure 8.37 considers the combined effect of the expansion and the liner.
% In Problem 4, the effect of the expansion was calculated, so including it
% here would include its effect twice.  Therefore, with an m of 1, there is
% not additional attenuation with the linear.
%
% The second case is to consider the m value of 2.7, which provides
% additional attenuation.  From Figure 8.37, the total attenuation of
% the lining is 10 dB.
%attenuation_rate = 10 / 0.127;  % 78.4  dB per meter



\subsection*{Problem 5b}

$l$, the liner thickness, is 0.0381 $m$.  $h$, the liner half width, is 0.0113 $m$.


\subsection*{Problem 5c}

The liner thickness ratio is 3.385 [unitless].  The normalized frequency is 0.0617 [unitless] for 940 Hz.


\subsection*{Problem 5d}

With a liner thickness ratio of 3.385, the left-side, middle attenuation rate curve is applicable.

With $\frac{l}{h}$ of 4, curve 5 is used




\subsection*{Problem 5e}

The flow resistivity, $R_1$, is about 1.74e5 $\frac{kg}{m^2 \cdot s}$ or rayl.






\newpage
\section{Appendix - Matlab Code for Problem 1}
\label{appendix:problem1}

\lstinputlisting[style=Matlab-Pyglike, basicstyle=\fontfamily{pcr}, numbers=left, keepspaces, mlshowsectionrules, basicstyle=\scriptsize]{../ACS_597_Module_1_Question_1_Monday_January_13_2025.m}



\newpage
\section{Appendix - Matlab Code for Problem 2}
\label{appendix:problem2}

\lstinputlisting[style=Matlab-Pyglike, basicstyle=\fontfamily{pcr}, numbers=left, keepspaces, mlshowsectionrules, basicstyle=\scriptsize]{../ACS_547_Module_1_Question_2_Monday_January_27_2025.m}



\newpage
\section{Appendix - Matlab Code for Problem 3}
\label{appendix:problem3}

\lstinputlisting[style=Matlab-Pyglike, basicstyle=\fontfamily{pcr}, numbers=left, keepspaces, mlshowsectionrules, basicstyle=\scriptsize]{../ACS_547_Module_1_Question_3_Wednesday_January_29_2025.m}



\newpage
\section{Appendix - Matlab Code for Problem 4}
\label{appendix:problem4}

\lstinputlisting[style=Matlab-Pyglike, basicstyle=\fontfamily{pcr}, numbers=left, keepspaces, mlshowsectionrules, basicstyle=\scriptsize]{../ACS_547_Module_1_Question_4_Wednesday_January_29_2025.m}



\newpage
\section{Appendix - Matlab Code for Problem 5}
\label{appendix:problem5}

\lstinputlisting[style=Matlab-Pyglike, basicstyle=\fontfamily{pcr}, numbers=left, keepspaces, mlshowsectionrules, basicstyle=\scriptsize]{../ACS_547_Module_1_Question_5_Friday_January_31_2025.m}







\end{document}






% Reference(s):

% https://tex.stackexchange.com/questions/180222/how-to-change-font-size-for-specific-lstlisting


































